Afterwards, the test phase was performed. The test set used, composed by the last 100 samples, 
was not voxelized in order to test model performance on data in its original format.
To be able to achieve this, ad-hoc backprojection functions were implemented.
Given a pointcloud, these functions attribute directly to each point the predicted label of the
corresponding voxel in the voxelized pointcloud (voxel grid).
With these operations, it was possibile to build the confusion matrix shown in figure \ref{fig:matrix}.
Using the data in this matrix, we were able to compute the following performance metrics over the 26 classes: 
mean accuracy, mean precision, mean recall and consequently F1-score.
Also, the average time needed by the model to perform a complete prediction (\textit{Tot. time}) was measured along the average partials relative to
pointcloud voxelization (\textit{Vox. time}), model prediction on the voxelgrid (\textit{Pred. time}) and
label backpropagation (\textit{Back. time}). The times have been taken over 30 random pointclouds predictions.
These two sets of data are reported in table \ref{tab:metrics}.

From these results, the following observation can be made.
Firstly, only a few classes appear among test set pointclouds.
Secondly, it can be noted from the confusion matrix and guessed by the performance metrics, 
that our model is able to correctly classify the most popular classes such as \textit{car},
\textit{road}, \textit{sidewalk}, \textit{building} and \textit{vegetation} 
but lacks in performance for the others. 
This happens because the latters appear a significantly lower amount of times and are possibly lumped together in a differently labeled voxel. 
This fact is reflected in the recall value, as two out of the classes present in the dataset 
are never predicted (yielding each one zero recall) causing the mean metric to be much lower than the other ones.
However, the obtained results are still acceptable thanks to the great unbalance among point classes.
Lastly, the total prediciton time of our model is particularly high,
making it unsuitable for real-time applications. Although, it must be noted that the "true" prediciton time
of \textit{VUE-net} is of only $11ms$ and the phases that generate the overhead in the total time
are the voxelization and backprojection ones. Thefore, for a possibile deployment of this model, only the 
implementation of these two operations shall be changed.

Using other Open3d functions, we were able to visualize the pointclouds with different colors depending
on their labels. Figure \ref{fig:visual} shows one of the pointclouds with the
correct colors (true labels) and the ones choosen by \textit{VUE-net} (predicted labels).
The observations made previously on performance agree with what can be seen from the images.
In fact, the model correcly predicts road, sidewalks, cars, buildings and vegetation but is not
able to recognize smaller, unpopular areas as the one in orange that corresponds to a fence.
