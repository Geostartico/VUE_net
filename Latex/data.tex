To build and train our network we used 1000 samples taken from the SemanticKITTI dataset \cite{behley2019iccv}.
The data is in the form of pointclouds. We used the
functions provided by the colab notebook on PointNet for reading and sampling the pointclouds. It must be noted that these function combined
do not read the full pointcloud but return a random subset of points of fixed size (20000 points).\par
Although, we decided not to use these sets of points as input for our model. As shown by PointNet,
a model based on pointclouds is particularly heavy and therefore unusable for real-time applications.
We instead decided to perform voxelization on the pointclouds and therefore our model's input is a grid of voxels.
A voxel is a volumetric unit of space, so it can be considered as a pixel in 3D space.
In order to transform the data we used the Open3D library api \cite{zhou2018open3d}. 
We decided to use voxelgrids of size $32^3$ to have simple calculations, as it is a power of $2$, and at the same time a manageable number of voxels to process.
The voxels are encoded into a $(32\times32\times32\times1)$ tensor, where the element $(X,Y,Z,0)$ is set to $1$ if and only if there's at least a point in the pointcloud that corresponds to the voxel with coordinates $(X,Y,Z)$, or 0 otherwise.\par
For the labels encoding we decided to use $(32\times32\times32\times\#classes)$ tensors which were generated as follows.
For each point in the pointcloud, the value of the tensor 
$labels(X,Y,Z,:)$ is updated as $labels(X,Y,Z,:)\textrm{+=}l$, where $(X,Y,Z)$ are the coordinates of the corresponding voxel and $l$ is the one-hot encoded version
of the point's label.
After iterating over all points labels, the obtained tensor was normalized along the appropriate dimension in order to obtain a probability distribution over the possibile classes for each voxel. 
As for the empty voxel, they were assigned to an additional (w.r.t. SemanticKITTI) fictitious class with total probability to 
allow our model to predict a voxel as empty.
This encoding was initially performed at run-time but it introduced a considerable over-head in the prediction time. 
Our tests showed that it takes about $1s$ in order to convert the pointcloud
using the adopted Open3D function, and $2s$ at most to convert the labels. This amount
is also not ideal to have at training time as it would take the most time out of a training round. 
To solve this issue we resorted to creating a new version of the dataset by computing and saving the voxelgrids, so that to access the voxelized samples
at training time, only their loading is needed. This also allowed to have a stable dataset, since the sampling of the pointclouds
is not deterministic. Although, in real-time applications this process must be performed on-the-fly and so, to assure a fast response,
the transformation should be performed at a lower level with respect to our implementation.